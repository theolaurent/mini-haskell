\documentclass[a4paper]{article}

\usepackage[utf8]{inputenc}
\usepackage[T1]{fontenc}
\usepackage[french]{babel}
\usepackage{listings}
\usepackage{examplep}
\usepackage{nameref}

\title{TODO}

\author{Jun \textsc{Maillard} et Théo \textsc{Laurent}}


\def\mlf{$ML^F$}

\begin{document}


\maketitle

% TODO : un anexe ``compilation'' avec l'utilisation de make + les
% dépendances (ocaml 4.02 et ocamlgraph

\section{Interface du compilateur}

\subsection{Compilation}
\PVerb{petitghc} est écrit en OCaml $4.02$. Il dépend de la bibliothèque ocamlgraph.

La commande \PVerb{make} compile \PVerb{petitghc}.
La commande \PVerb{make clean} supprime tous les fichiers générés lors de la compilation.

\subsection{Arguments}
En plus des options \PVerb{--parse-only} et \PVerb{--type-only}, les options suivantes sont disponibles :
\begin{itemize}
\item[\PVerb{--print-ast}] Affiche l'arbre de syntaxe abstraite parsé.
\item[\PVerb{--print-type}] Affiche le type de chacune des variables globales.
\end{itemize}

\subsection{Report d'erreurs}
Lors de chaque phase de la compilation (en l'occurence l'analyse syntaxique et le typage), le compilateur tente de trouver le plus d'erreurs possibles. Si à la fin d'une phase des erreurs ont été trouvées, elles sont affichées (dans l'ordre d'occurence) et la compilation s'arrête avec le code d'erreur $1$.

Au cours du typage, lorsqu'une unification échoue, le terme de l'ast en cours d'inférence puis la trace des unifications successives sont affichées.

\section{Choix techniques}
\subsection{Analyse syntaxique}
Le lexeur et le parseur sont générés à l'aide des outils ocamllex et menhir, respectivement.

L'arbre syntaxique produit est un type riche : on distingue notamment les opérations binaires primitives (arithmétiques, logique, de comparaison et cons) des autre variables. Par contre les primitives \PVerb{div}, \PVerb{rem}, \PVerb{putChar} et \PVerb{error} sont traités comme n'importequ'elle variable.

\subsection{Typage}
Pour le typage, un système de type \og à la \mlf \fg ~a été implémenté
(cf. section \nameref{mlf}). % TODO : utiliser les references Latex

\subsection{Génération de code}
Pour la génération du code le compilateur passe par une représentation intermédiaire (IR).

Cette représentation est une suite d'instruction linéaire. Le résultat de chaque instruction de l'IR est supposé être passé implicitement à l'instruction suivante.

Elle peut accéder à la pile via les instructions \PVerb{Push} et \PVerb{Pop} ets certaines instructions modifient implicitement l'état de la pile (\PVerb{ApplyUncons} ajoute deux élément à la pile ; \PVerb{ApplyUnnop}, \PVerb{ApplyBinop} et \PVerb{CallFun} enlèvent des élements de la pile (leur argument) ; \PVerb{StartFun} et \PVerb{Return} initialise et restaure le tableau d'activation lors d'un appel).

Lors de la transformation de l'ast vers cette IR, les noms de
variables sont remplaçés par leur emplacement (locale, argument, dans
l'environement la fermeture, globale).

C'est également a ce moment là que la construction des fermetures est
explicité, ainsi que la congélation et l'évaluation forcée. Sont
congelées les arguments des fonctions avant application et les valeurs
liées par un let. Sont forcés : les arguments des primitives
artihmétiques, logiques, etc... ; les arguments des primitives
\PVerb{div}, \PVerb{rem}, \PVerb{putChar} et \PVerb{error} lors de
l'application du dernier argument ; les conditions des instructions if
et enfin l'expression filtrée par \PVerb{case ... of}.


\section{Difficultés rencontrées}

\subsection{Choix sur la forme de l'arbre de syntaxe abstraite}
Nous avions choisis, car cela simplifiait l'algorithme de typage,
d'utiliser un AST minimaliste. Il contenait seulement l'application à
un arguemnt, l'abstraction d'une variable et la construction \og let
\fg, le reste étant abstrait via un ensemble de primitives (la
condition \og if \fg, le pattern-matching, etc...).

Ce choix s'est avéré très peu pratique pour la gérération de code où
nous avions besoin d'un maximum de précision pour pouvoir procéder à
des optimisations. Nous sommes donc revenu sur notre choix et avons
intégré à l'AST des \og constructions spéciales \fg~puis des opérateurs
binaires natifs.

Celà a grandement complexifié le code de l'algorithme d'inférence de type.

\subsection{Calcul des variables libres lors du typage}

Lors du typage, le calcul des variables (de type) libres était
extrêmement lent (plusieurs heures sur certains des exemples fournis).
Nous avons donc choisis de stocker les variables libres contenues dans
un schéma ou un type et de seulement les mettre à jour à la volée.


\subsection{Définitions mutuelement récursives}

La compilation des valeurs mutuellement récursives a posé quelques
petites difficultés, nous nous sommes rendu compte qu'il était
nécéssaire d'allouer l'espace nécessaire pour chacune des valeurs
\emph{avant} la compilation de celles-ci, pour connaitre leurs
emplacements dans le tas et pouvoir les ajouter à leurs environements
respectifs.

\section{Extension : \mlf}\label{mlf}

\subsection{Un système de type intéressant}

Nous avons implémenté le système de type décrit dans la thése de
Didier le Botlan \textit{TODO : référence + bibliographie}. Il s'agit
d'un algorithme d'inférence de type \og à la ML \fg~pour le système F.
L'inférence de type dans le systeme F étant indécidable, des
annotations sont parfois nécessaires pour obtenir des types polymorphes
d'ordre supérieur (cependant l'instanciation des schémas de type est
toujours implicite). La puissance du système \mlf réside dans le fait
que tout les programme (non annotés) typables dans ML sont typables
dans \mlf \emph{sans annotations}.

\subsection{Exemples d'utilisations}

\subsubsection{Typage du combinateur $\omega$}
Il possible de donner divers type à l'expression $(\lambda x.\ x x)$ dans le système F. Le fichiers \og tests/supp/mlf.hs \fg{} illustre qu'il est typable avec \mlf en annotant l'argument $x$ par le schéma $\forall \alpha, \alpha \rightarrow \alpha$.


\subsubsection{Les paires du $\lambda$-calcul}
Le système F permet aussi d'obtenir un encodage des paires similaire à celui du $\lambda$-calcul. Le fichier \og tests/supp/pair.hs \fg{} présente cet encodage.

\subsubsection{Types existentiels}
L'encodage des types existentiels est réalisable en annotant convenablement les paires. Ils permettent d'utiliser de manière uniforme des valeurs de type distincts. Un tentative d'encodage se trouve dans le fichier \og tests/supp/existentiel.hs \fg.

\end{document}
