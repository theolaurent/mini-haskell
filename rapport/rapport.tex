\documentclass[a4paper]{article}

\usepackage[utf8]{inputenc}
\usepackage[T1]{fontenc}
\usepackage[french]{babel}
\usepackage{listings}

\begin{document}

\section{Interface du compilateur}

\subsection{Arguments}
En plus des options \texttt{-{}-parse-only} et \texttt{-{}-type-only}, les options suivantes sont disponibles :
\begin{itemize}
\item[\texttt{-{}-print-ast}] Affiche l'arbre de syntaxe abstraite parsé.
\item[\texttt{-{}-print-type}] Affiche le type de chacune des variables globales.
\end{itemize}

\subsection{Report d'erreur}
Lors de chaque phase de la compilation (en l'occurence l'analyse syntaxique et le typage), le compilateur tente de trouver le plus d'erreurs possibles. Si à la fin d'une phase des erreurs ont été trouvées, elles sont affichées (dans l'ordre d'occurence) et la compilation s'arrête avec le code d'erreur $1$.

\subsection{Erreur de typage}
Lorsqu'une unification échoue, le terme de l'ast en cours d'inférence puis la trace des unifications successives sont affichées.

\section{Choix techniques}
\subsection{Analyse syntaxique}
Le lexeur et le parseur sont générés à l'aide des outils ocamllex et menhir, respectivement. \\
L'arbre syntaxique produit est un type riche : on distingue notamment les opérations binaires primitives (arithmétiques, logique, de comparaison et cons) des autre variables. Par contre les primitives \texttt{div}, \texttt{rem}, \texttt{putChar} et \texttt{error} sont traités comme n'importequ'elle variable.

\subsection{Typage}
Pour le typage, un système de type "à la MLF" a été implémenté (cf. section "Extension : MLF").

\subsection{Génération de code}


\section{Difficultés rencontrées}

\section{Extension : MLF}


\end{document}
